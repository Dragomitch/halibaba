\documentclass[11pt,a4paper]{report}
\usepackage[T1]{fontenc}
\usepackage[utf8]{inputenc}
\usepackage{lmodern}
\usepackage[french]{babel}
\usepackage{fullpage}
\usepackage{listings}
\usepackage{graphicx}


\title{Projet de SQL: Game of Trones}
\author{Jeremy Wagemans \and Philippe Dragomir}
\date{\today}

\begin{document}
    \lstset{
        tabsize=2,
        basicstyle=\footnotesize,
        frame=single,
        breaklines=true,
        literate=
            {é}{{\'e}}1
            {è}{{\`e}}1
            {ô}{{\^o}}1
            {ê}{{\^e}}1
            {ç}{{\c{c}}}1
            {à}{{\'a}}1,
    }

\maketitle

\addcontentsline{toc}{chapter}{Table des mati\`eres}
\tableofcontents

\begingroup
\setlength{\parskip}{\baselineskip}
\chapter{Introduction}

Afin d’appliquer les méthodologies et les notions enseignées aux cours de base de données, nous avions pour objectif de réaliser, par groupe de deux, une application de gestion des demandes.
\par

En effet, présentation...
\par

Au terme du projet, nous avons donc du délivrer une solution en parfaite adéquation avec un cahier de répartition des charges et répondant à des critères de qualité stricts. Ce rapport permet donc d’exposer de manière précise son fonctionnement ainsi que certaines. Il est structuré comme suit.
\par

Dans un premier temps, nous développerons...
\par

Ensuite,
\par

Enfin, nous proposerons le code source des deux applications développées.
\par

\endgroup

\setlength{\parskip}{0pt}

\chapter{Présentation de la solution}

\chapter{Base de données}

\section{Script d'installation}
\lstinputlisting[language=sql]{../SQL/install.sql}
\newpage

\section{Script d'insertion de données valides}
\lstinputlisting[language=sql]{../SQL/src/insert_valid_data.sql}
\newpage

\section{Script d'insertion de données invalides}
\lstinputlisting[language=sql]{../SQL/src/insert_invalid_data.sql}
\newpage

\chapter{Application java}
\section{App.java}
\lstinputlisting[language=java]{../app/marche_halibaba/src/marche_halibaba/App.java}
\newpage

\section{ClientsApp.java}
\lstinputlisting[language=java]{../app/marche_halibaba/src/marche_halibaba/ClientsApp.java}
\newpage

\section{HousesApp.java}
\lstinputlisting[language=java]{../app/marche_halibaba/src/marche_halibaba/HousesApp.java}
\newpage

\section{Utils.java}
\lstinputlisting[language=java]{../app/marche_halibaba/src/marche_halibaba/Utils.java}
\newpage

\chapter{Conclusion}

A l’issue d’un mois de travail intensif, nous pouvons affirmer que ce projet s’est terminé sans encombre et dans les délais.
Nous avons atteint les objectifs que nous nous sommes fixés initialement et avons réalisé une solution répondant parfaitement au cahier des charges.
\par

Nous estimons la période de réalisation de l’entièreté de l’application à 50 heures réparties comme suit:
5h pour l’analyse, 30 heures pour la conception de la base de données et 15h pour le développement de l’application java.
\par

Nous avons eu l’opportunité, grâce à ce projet, d’améliorer et d’approfondir nos connaissances en SQL ainsi qu’à nous familiariser aux bonnes pratiques de jdbc.
Nous avons également pu appliquer l’ensemble des savoir-faire acquis en cours de conception de bases de données.
\par

Du point de vue humain, il nous a permis d’apprendre à mieux nous connaître.
Nous avons appris à travailler ensemble de manière efficace en répartissant la charge de travail selon nos forces et faiblesses.
\par

C’est donc pleinement satisfaits que nous délivrons ce projet aujourd’hui.

\end{document}
